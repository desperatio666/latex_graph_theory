\documentclass{beamer}
\usetheme{Boadilla}
\usepackage[utf8]{inputenc}
\usepackage{amsmath, amssymb}
\usepackage{tikz}
\usepackage{hyperref} %%%%TODO: anschauen wie man hyperrefs macht lol%%%%
\usetikzlibrary{external}
\tikzexternalize[prefix=tikz/]

\title{Graphentheorie} %TODO: maybe think of a more interesting title lol
\author{F. Butter }
\date{25/04/2022}
\institute{Technische Universität Dresden}
\begin{document}

\maketitle

\begin{frame}
    \frametitle{Übersicht}
    \tableofcontents
        \section{Was sind Graphen? - Grundbegriffe, Überblick}
        \label{sec:one}
        \section{Gimme five! (Handshaking-Lemma)}
        \label{sec:two}
        \section{Jeder kennt jeden - vollständige Graphen}
        \label{sec:three}
        \section{Was sind Wege?}
        \label{sec:four}
        \section{Wozu das Ganze? (Ausblick, Anwendungen)}
        \label{sec:five}
\end{frame}

\begin{frame}
    \frametitle{\hyperref[sec:one]{1. Was sind Graphen? (Grundbegriffe, Überblick)}}
        \textbf{Definition.} 
            Ein \textbf{Graph} \( G = (V, E)\) besteht aus einer nichtleeren Menge V und einer (möglicherweise leeren) Kantenmenge \(E\), wobei jede Kante \(e \in E\) zwei Knoten \(u, v\) miteinander verbindet.\\
    \bigskip
\end{frame}

\begin{frame}
    \begin{itemize}
        \item Wenn \(u = v\) für eine Kante \(e = \{u, v\}\) gilt, so nennt man dies eine \textit{Schlinge}. 
        \item Wenn für zwei Kanten \(e, f \in E\) gilt: \(e = f = \{u , v\}\) (gleiche Endknoten), so sind diese \textit{parallel}. 
        \item Ein Graph heißt \textbf{einfach}, genau dann wenn er weder Schlingen noch parallele Kanten enthält.
    \end{itemize}
\end{frame}

\begin{frame}
    \textbf{Definition.}
    Sei \(G = (V, E)\) ein Graph, \(u, v \in V\) und \(e \in E\). 
    Es gilt jeweils:
    \begin{itemize}
        \item \textbf{\textit{inzident}}: 
            Ein Knoten \(v\) und eine Kante \(e\) \textit{inzidieren} miteinander, wenn \(v\) ein \textit{Endknoten} von \(e\) ist.
        \item \textbf{\textit{Nachbar eines Knotens}}:
            Gilt für eine Menge an Knoten \(\{u, v\} \in E\), so sind \(u\) und \(v\) \textit{adjazent} bzw. \textit{benachbart} in \(G\) und heißen \textit{Nachbarn}. 
        \item \textbf{\textit{Nachbar einer Kante}}:
            Zwei Kanten \(e = \{u, v\} \in E\) und \(f = \{v, w\} \in E\) heißen \textit{adjazent/benachbart}, wenn sie einen gemeinsamen Knoten haben.
    \end{itemize}
\end{frame}

\begin{frame}{1.? Konventionen}
    \begin{itemize}
        \item Bezeichne eine Kante \(e\) mit \(e:= \{u, v\}\) (was wir NICHT verwenden: \(e = (u, v)\) , \(e = uv\) )
        \item \textit{Endknoten.} Bezeichne einen Endknoten einer Kante \(e\) als ein Knoten \(u \in V\), der in einer Kante \(e = \{u, v\}\) enthalten ist. %%%%maybe tikZ img to illustrate%%%
        \item Knotenanzahl in einem Graphen: \(n\)
        \item Kantenanzahl in einem Graphen: \(m\)
        \item Wir beschäftigen uns von nun an mit \textit{einfachen} Graphen.
    \end{itemize}
%%%%%TODO: irgendwelche TikZ bilder einfügen oder so lol idk%%%%%
\end{frame}
\begin{frame}
    \hyperref[sec:two]{\frametitle{2. Gimme five! (Handshaking-Lemma)}} %%%TODO: HOW DO HYPERREFS WORK PROPERLY????%%%%
        \textbf{Definition.} Sei \(G = (V, E)\) ein Graph. Der \textbf{Grad} \(d(v)\) eines Knotens \(v \in V\) ist die Anzahl des Auftretens von \(v\) als Endknoten einer Kante. Wenn \(G\) einfach ist, so ist \(d(v)\) gleich der Anzahl der Nachbarn von v. \\
        \bigskip
        Für einen Knoten \(v \in V\), bezeichne
        \begin{itemize}
            \item \(\delta(G)\) als \textit{Minimalgrad}
            \item \(\Delta(G)\) als \textit{Maximalgrad}
        \end{itemize}
        des Graphen.
\end{frame}

\begin{frame}
    \textbf{Lemma.} \textit{(Handshaking-Lemma.)}
    Die Summe der Knotengrade entspricht zweimal der Anzahl der Knoten, also
    \[\sum_{v \in V}d(v) = 2 \cdot |E| = 2m.\]
\end{frame}

\begin{frame}
    \textbf{Lemma.}
    In einem Graphen ist die Anzahl der Knoten mit \textit{ungeradem Grad} gerade.
\end{frame}

\begin{frame}
    \frametitle{3. Jeder kennt jeden -- vollständige Graphen}
    \textbf{Definition.}
    Ein Graph \(G = (V, E)\) heißt \textit{vollständig}, wenn jeder Knoten zu jedem anderen benachbart ist.\\
    \bigskip
    \(\xrightarrow[]{}\) da \(G\) nur von \(n\) abhängig, bezeichne \(G\) als \(K_n\)\\
    \(\xrightarrow[]{} \delta(G) = \Delta(G)\) 
    %%%%maybe some nice tikZ imgs once again%%%%
\end{frame}

\begin{frame}
    \textbf{Definition.} Ein Graph \(G = (V, E)\) heißt \textit{bipartit}, wenn seine Knotenmenge in zwei Teilmengen an Knoten \(A, B \in V\) so zerlegt werden können, dass nur Kanten \textit{zwischen}, aber nicht \textit{innerhalb} einer Knotenmenge verlaufen.\\ 
    \bigskip
    \(\xrightarrow[]{}\) falls ein Graph die \textit{maximale Anzahl an Kanten} \(m_{\text{max}} = |A| \cdot |B|\) besitzt, nennt man ihn \textbf{vollständig bipartit}
\end{frame}

\begin{frame}{4. Was sind Wege?}
    \textbf{Definition.} (\textit{Teilgraphen.}) Sei \(G = (V, E)\) ein Graph. Ein Graph \(G' = (V', E')\) ist ein \textit{Teilgraph} von G, wenn \(V' \subseteq V\) und \(E' \subseteq E\) gilt.\\
    Falls E' jede Kante aus E enthält, die zwei Knoten aus V' verbindet, so sagt man: G' ist ein von \(V' \subseteq V\) \textit{induzierter Teilgraph} bzw. \(Untergraph\) von G.
\end{frame}

\begin{frame}
    \textbf{Definition.} Ein \textbf{Kantenzug \(Z\)} in \(G\) ist eine Folge von Knoten und Kanten in \(G\), die sich wie folgt hintereinanderschreiben lässt:
    \[Z = x_0e_0x_1e_1x_2e_2\cdots x_{k-1}e_{k-1}x_k\]   
    wobei \(e_i = \{x_i, x_{i+1}\}\) gilt.\\
    \bigskip
    %%%%%TODO: another tikZ iMaGe von nem Kantenzug!%%%%%
    %%%%%TODO: evtl. andere Def.: Z = e_0e_1e_2...e_k, aber Alternative ist Buch-Def.%%%%%
\end{frame}

\begin{frame}
    \textbf{Definition.}
    Sei \(Z\) ein Kantenzug. Definiere nun \(Z\) als...
    \begin{itemize}
        \item einen \textbf{Weg}: \\
        Alle Kanten in \(Z\) sind verschieden. Die Knoten \(x_0\) und \(x_k\) sind die \textit{Endknoten} eines Weges \(p\). 
        %%%%TODO: definiere Länge eines Weges%%%%
        \item einen \textbf{Kreis}: \\
        Ein \textbf{Kreis} ist ein Weg, bei dem gilt: \(x_0 = x_k\) (gleicher Anfangs- und Endknoten).
    \end{itemize}
\end{frame}

\begin{frame}{Es könnte alles so einfach sein...}
    \textbf{Definiere nun:}\\
    \begin{itemize}
        \item \textbf{einfacher Weg.}\\
        Ein Weg heißt \textit{einfach}, wenn keine doppelten Knoten vorkommen, also kein Knoten doppelt "durchlaufen" wird.
        \item \textbf{einfacher Kreis.}\\
        Ein Kreis heißt \textit{einfach}, wenn außer dem Endknoten kein Knoten doppelt vorkommt. 
    \end{itemize}
    %%%%TODO: blablabla tikZ imgs blablabla%%%%
\end{frame}

\begin{frame}
    \textbf{Satz.} \textit{(Einfache Wege und einfache Kreise.)} Jeder Graph \(G\) enthält einen einfachen Weg der Länge \(\delta(G)\). Falls \(\delta(G) \geq 2\) gilt, dann enthält er auch einen einfachen Kreis von mindestens der Länge \(\delta(G) + 1\). (\(\delta(G)\) ist der Minimalgrad.)
    %%%%TODO: blablabla proof blablabla%%%% or maybe example of what is meant
\end{frame}

\begin{frame}{4.? Zusammenhänge(nde Graphen)}
    \textbf{Definition.} Ein Graph G heißt \textbf{zusammenhängend}, wenn zu je zwei Knoten \(u, v \in V\) von \(G\) ein \textit{Weg} von \(u\) nach \(v\) existiert.\\
    Nicht zusammenhängende Graphen heißen \textbf{unzusammenhängend}.\\
    Eine \textbf{Zusammenhangskomponente} \(H\) von \(G\) ist ein \textit{maximal zusammenhängender Teilgraph} von \(G\). 
    
\end{frame}

\begin{frame}{5. Wozu das Ganze? (Ausblick, Anwendungen)}
    %%%% kürzeste-Wege-Probleme: Wie ermittelt Navi kürzesten Weg?
    %%%% Färbungsprobleme: Graphen unterschiedlich färben (wie bei Würfel) -> Stundenpläne (Knoten Kombi aus Dozent & Kurs (Fr Müller Klasse 7 & 8 -> 2 Knoten), Kanten wenn 2 Veranstaltungen nicht gleichzeitig sind, Färbung d. Knoten -> benachbarte Knoten unterschiedlich gefärbt -> Ziel: möglichst wenig Farben)
    %%%% Rundreise: Kreise, der alle Knoten "besucht" 
    %%%%TODO: not sure what to write here yet%%%%
    
\end{frame}
\begin{frame}
    \frametitle{Anwendungen}
    \begin{itemize}
        \item Das Internet!\\
        \begin{itemize}
            \item Knoten: einzelne Website
            \item Kanten: Hyperlink zwischen den Websites (\(=\) Links)
        \end{itemize}
        \item Chemie (\(\xrightarrow[]{}\) chemische Graphentheorie)\\
        \begin{itemize}
            \item Knoten: Atome
            \item Kanten: Bindungen zwischen den Atomen 
        \end{itemize}
        \item Grammatik von Sprachen 
        \item usw. \(\cdots\)
        \item generell überall in der Wissenschaft, v.a. Informatik
    \end{itemize}
\end{frame}

\end{document}

%%%%%%%%%%%%%%%%%%%%%%%%%%%%%%%%%%
%template for frames:
%\begin{frame}
%    \frametitle{}
%\end{frame}