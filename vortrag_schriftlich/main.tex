\documentclass{article}
\usepackage[utf8]{inputenc}
\usepackage{amsmath, amssymb}

\title{Graphentheorie}
\author{Franziska Butter}
\date{September 2022}
\institute{Technische Universität Dresden}

\begin{document}

\maketitle

\section{Übersicht}
	\tableofcontents

\newpage

\section{Was sind Graphen? - Eine Einführung}
In diesem Paper möchte ich eine erste Einführung zum Thema Graphentheorie geben. Wer noch nie mit dem Thema in Berührung gekommen ist, soll nach dem Studieren dieser Einführung einen grundlegenden Überblick über die Basics bekommen. Hierzu beginnen wir mit den gundlegenden Definitionen, werden uns einsteigende Beispiele anschauen, sowie ein paar erste Sätze.\\
\bigskip
\textbf{Definition.} 
            Ein \textbf{Graph} \( G = (V, E)\) besteht aus einer nichtleeren Menge V und einer (möglicherweise leeren) Kantenmenge \(E\), wobei jede Kante \(e \in E\) zwei Knoten \(u, v\) miteinander verbindet.\\
\bigskip
Beginnen wir also direkt mit dieser Definition. Zunächst abstrakt wirkend, schauen wir uns doch erst einmal ein Beispiel an:\\
Angenommen, Katrin, Tom, Erik und Lisa sind vier Studierende. Katrin kennt Tom, Lisa und Erik, Lisa kennt Tom und Katrin, Tom kennt Lisa und Katrin, und Erik kennt Katrin. Zusätzlich kennt jeder natürlich noch sich selbst.\\
Diese Freundschaften kann man nun in einem mathematischen Graphen betrachten. Jeder Studierende ist hierbei ein Knoten \(v\) , und jede Freundschaft zwischen zwei Knoten bildet eine Kante \(e\). Zusammen bilden alle Knoten die Knotenmenge \(V\), und alle Kanten die Kantenmenge \(E\).\\
Zusammen bilden \((V, E)\) den Graphen \(G\), welchen man graphisch darstellen kann.\\
\bigskip
Nun könne die Knoten und Kanten jeweils verschiedene Beziehungen zueinander haben.\\
Um auf unser Beispiel zurückzukommen: Eine triviale, jedoch erwähnenswerte Erscheinung ist noch, dass jeder sich selbst kennt. Übersetzt in die Sprache der Graphentheorie bedeutet dies, dass eine Kante den gleichen Endpunkt hat. Dieses Phänomen nennt man \textbf{Schlinge}.
\bigskip
Wenn man (redundanterweise) die Information "Tom kennt Lisa und Lisa kennt Tom" angeben möchte, könnte man auch vom Knoten "Tom" eine Kante zum Knoten "Lisa" ziehen, und vom Knoten "Lisa" eine Kante zum Knoten "Tom". Wenn also zwei Kanten den gleichen Endpunkt haben, so nennt man diese \textit{parallel}.\\
\bigskip
Diese Begriffe nochmal in Definitionen zusammengepackt:\\
\begin{itemize}
	\item Wenn \(u = v\) für eine Kante \(e = \{u, v\}\) gilt, so heißt diese Kante \textit{Schlinge}.
	\item Wenn für zwei Kanten \(e, f \in E\) gilt: \(e = f = \{u, v\}\) (gleiche Endknoten), so heißen diese \textit{parallel}. 
	\item Ein Graph heißt \textit{einfach}, genau dann, wenn er weder Schlingen noch parallele Kanten enthält.
\end{itemize}
Unser Beispielgraph mit den Studierenden, welche untereinander befreundet sind, ist hier einfach, da die Informationen, welche Schlingen und parallele Kanten bilden würden, hier redundant, beziehungsweise nicht notwendig sind.\\

\nepwage
Nun wird es zunächst wieder etwas technischer. Wir schauen uns jetzt weiter Beziehungen von Knoten und Kanten untereinander an:\\
\textbf{Definitionen.}
    Sei \(G = (V, E)\) ein Graph, \(u, v \in V\) und \(e \in E\).
    Es gilt jeweils:
    \begin{itemize}
        \item \textbf{\textit{inzident}}:
            Ein Knoten \(v\) und eine Kante \(e\) \textit{inzidieren} miteinander, wenn \(v\) ein \textit{Endknoten} von \(e\) ist.
        \item \textbf{\textit{Nachbar eines Knotens}}:
            Gilt für eine Menge an Knoten \(\{u, v\} \in E\), so sind \(u\) und \(v\) \textit{adjazent} bzw. \textit{benachbart} in \(G\) und heißen \textit{Nachbarn}.
        \item \textbf{\textit{Nachbar einer Kante}}:
            Zwei Kanten \(e = \{u, v\} \in E\) und \(f = \{v, w\} \in E\) heißen \textit{adjazent/benachbart}, wenn sie einen gemeinsamen Knoten haben.
    \end{itemize}
Diese Definitionen macht man sich am Besten klar, indem man sich selbst einmal einen Graphen anschaut (dafür eignet sich beispielsweise unser einführendes Beispiel mit den Studierenden).\\

\subsection{Konventionen}
Bevor wir uns weiter in der Welt der Graphentheorie hervortasten, zunächst ein paar Konventionen, die wir festlegen:
\begin{itemize}
        \item Wir bezeichnen eine  Kante \(e\) hier mit \(e:= \{u, v\}\). Es gibt auch noch Alternativbezeichnungen für Kanten. Was wir NICHT verwenden: \(e = (u, v)\) , \(e = uv\).
        \item \textit{Endknoten.} Wir bezeichnen einen Endknoten einer Kante \(e\) als ein Knoten \(u \in V\), der in einer Kante \(e = \{u, v\}\) enthalten ist. 
        \item Knotenanzahl in einem Graphen: \(n\)
        \item Kantenanzahl in einem Graphen: \(m\)
        \item Wir beschäftigen uns von nun an mit \textit{einfachen} Graphen.
    \end{itemize}

\newpage
\section{Gimme five! (Handshaking-Lemma)}
Nun, da die nötigen Grundlagen gelegt sind, kann es auch schon mit der ersten wichtigen Erkenntnis in der Graphentheorie, nämlich dem Handshaking-Lemma, weitergehen.\\
\bigskip
Bevor wir jedoch dazu kommen, müssen wir erneut eine relevante Definition einführen:\\
\bigskip
\textbf{Definition.} Sei \(G = (V, E)\) ein Graph. Der \textbf{Grad} \(d(v)\) eines Knotens \(v \in V\) ist die Anzahl des Auftretens von \(v\) als Endknoten einer Kante. Wenn \(G\) einfach ist, so ist \(d(v)\) gleich der Anzahl der Nachbarn von v.\\
\bigskip
Diese Definition lässt sich sehr einfach auf unseren Beispielgraphen vom Anfang übertragen: Um beispielsweise den Grad des Knotens "Katrin" zu bestimmen, zählen wir einfach, wie viele Kanten von dem Knoten ausgehen, also, mit wie vielen Studierenden Katrin aus der Gruppe von 4 Menschen befreundet ist, also die Anzahl der Nachbarn des Knoten "Katrin". In diesem Falle sind es \(3\) Kanten, also ist Katrin mit 3 Studierenden (Tom, Erik, Lisa) befreundet.\\
Das Zählen der Freunde funktioniert aber nur, da G einfach ist. Würde es beispielsweise noch eine Schlinge für "Studierender x kennt sich selbst" geben, so wäre der Knotengrad jeweils die Anzahl der Nachbarn der Knoten \(+1\).\\
\bigskip
Weiterhin bezeichnen wir für einen Knoten \(v \in V\):
\begin{itemize}
	\item \(\delta(G)\) als \textit{Minimalgrad} eines Graphen, und
	\ittem \(\Delta(G)\) als \textit{Maximalgrad} eines Graphen.
\end{itemize}
\textit{Frage an den geneigten Leser.}\\
Was sind jeweils der Minimal- und Maximalgrad des Beispielgraphen?\\
\bigskip
Nun können wir auch schon zur versprochenen wichtigen Erkenntnis kommen: Dem Handshaking-Lemma.\\
\textbf{Lemma.} \textit{(Handshaking-Lemma.)}
Die Summe der Knotengrade entspricht zweimal der Anzahl der Knoten, also
\[\sum_{v \in V}d(v) = 2 \cdot |E| = 2m.\]
\bigskip
\textbf{Beweis.} %%%TODO: Beweis einfügen%%%
%%%TODO: Zeugs aus neuer Version übernehmen :S %%%
\end{document}
