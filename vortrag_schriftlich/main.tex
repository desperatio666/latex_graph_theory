\documentclass{article}
\usepackage[utf8]{inputenc}
\usepackage{amsmath, amssymb}
\usepackage{amsthm}
\newtheorem{theorem}{Satz}
\newtheorem{definition}{Definition}
\newtheorem{lemma}{Lemma}

\title{Graphentheorie}
\author{Franziska Butter}
\date{September 2022}

\begin{document}

\maketitle

\section{Übersicht}
	\tableofcontents

\newpage

\section{Was sind Graphen? - Eine Einführung}
In diesem Paper möchte ich eine erste Einführung zum Thema Graphentheorie geben. Wer noch nie mit dem Thema in Berührung gekommen ist, soll nach dem Studieren dieser Einführung einen grundlegenden Überblick über die Basics bekommen. Hierzu beginnen wir mit den gundlegenden Definitionen, werden uns einsteigende Beispiele anschauen, sowie ein paar erste Sätze.\\
\bigskip
\begin{definition} 
            Ein \textbf{Graph} \( G = (V, E)\) besteht aus einer nichtleeren Menge V und einer (möglicherweise leeren) Kantenmenge \(E\), wobei jede Kante \(e \in E\) zwei Knoten \(u, v\) miteinander verbindet.
\end{definition}
\bigskip
Beginnen wir also direkt mit dieser Definition. Zunächst abstrakt wirkend, schauen wir uns doch erst einmal ein Beispiel an:\\
Angenommen, Katrin, Tom, Erik und Lisa sind vier Studierende. Katrin kennt Tom, Lisa und Erik, Lisa kennt Tom und Katrin, Tom kennt Lisa und Katrin, und Erik kennt Katrin. Zusätzlich kennt jeder natürlich noch sich selbst.\\
Diese Freundschaften kann man nun in einem mathematischen Graphen betrachten. Jeder Studierende ist hierbei ein Knoten \(v\) , und jede Freundschaft zwischen zwei Knoten bildet eine Kante \(e\). Zusammen bilden alle Knoten die Knotenmenge \(V\), und alle Kanten die Kantenmenge \(E\).\\
Zusammen bilden \((V, E)\) den Graphen \(G\), welchen man graphisch darstellen kann.\\
\bigskip
Nun könne die Knoten und Kanten jeweils verschiedene Beziehungen zueinander haben.\\
Um auf unser Beispiel zurückzukommen: Eine triviale, jedoch erwähnenswerte Erscheinung ist noch, dass jeder sich selbst kennt. Übersetzt in die Sprache der Graphentheorie bedeutet dies, dass eine Kante den gleichen Endpunkt hat. Dieses Phänomen nennt man \textbf{Schlinge}.
\bigskip
Wenn man (redundanterweise) die Information "Tom kennt Lisa und Lisa kennt Tom" angeben möchte, könnte man auch vom Knoten "Tom" eine Kante zum Knoten "Lisa" ziehen, und vom Knoten "Lisa" eine Kante zum Knoten "Tom". Wenn also zwei Kanten den gleichen Endpunkt haben, so nennt man diese \textit{parallel}.\\
\bigskip
Diese Begriffe nochmal in Definitionen zusammengepackt:\\
\begin{itemize}
	\item Wenn \(u = v\) für eine Kante \(e = \{u, v\}\) gilt, so heißt diese Kante \textit{Schlinge}.
	\item Wenn für zwei Kanten \(e, f \in E\) gilt: \(e = f = \{u, v\}\) (gleiche Endknoten), so heißen diese \textit{parallel}. 
	\item Ein Graph heißt \textit{einfach}, genau dann, wenn er weder Schlingen noch parallele Kanten enthält.
\end{itemize}
Unser Beispielgraph mit den Studierenden, welche untereinander befreundet sind, ist hier einfach, da die Informationen, welche Schlingen und parallele Kanten bilden würden, hier redundant, beziehungsweise nicht notwendig sind.\\

\newpage
Nun wird es zunächst wieder etwas technischer. Wir schauen uns jetzt weiter Beziehungen von Knoten und Kanten untereinander an:\\
\begin{definition}
    Sei \(G = (V, E)\) ein Graph, \(u, v \in V\) und \(e \in E\).
    Es gilt jeweils:
    \begin{itemize}
        \item \textbf{\textit{inzident}}:
            Ein Knoten \(v\) und eine Kante \(e\) \textit{inzidieren} miteinander, wenn \(v\) ein \textit{Endknoten} von \(e\) ist.
        \item \textbf{\textit{Nachbar eines Knotens}}:
            Gilt für eine Menge an Knoten \(\{u, v\} \in E\), so sind \(u\) und \(v\) \textit{adjazent} bzw. \textit{benachbart} in \(G\) und heißen \textit{Nachbarn}.
        \item \textbf{\textit{Nachbar einer Kante}}:
            Zwei Kanten \(e = \{u, v\} \in E\) und \(f = \{v, w\} \in E\) heißen \textit{adjazent/benachbart}, wenn sie einen gemeinsamen Knoten haben.
    \end{itemize}
\end{definition}
\bigskip
Diese Definitionen macht man sich am Besten klar, indem man sich selbst einmal einen Graphen anschaut (dafür eignet sich beispielsweise unser einführendes Beispiel mit den Studierenden).\\

\subsection{Konventionen}
Bevor wir uns weiter in der Welt der Graphentheorie hervortasten, zunächst ein paar Konventionen, die wir festlegen:
\begin{itemize}
        \item Wir bezeichnen eine  Kante \(e\) hier mit \(e:= \{u, v\}\). Es gibt auch noch Alternativbezeichnungen für Kanten. Was wir NICHT verwenden: \(e = (u, v)\) , \(e = uv\).
        \item \textit{Endknoten.} Wir bezeichnen einen Endknoten einer Kante \(e\) als ein Knoten \(u \in V\), der in einer Kante \(e = \{u, v\}\) enthalten ist. 
        \item Knotenanzahl in einem Graphen: \(n\)
        \item Kantenanzahl in einem Graphen: \(m\)
        \item Wir beschäftigen uns von nun an mit \textit{einfachen} Graphen.
    \end{itemize}

\newpage
\section{Gimme five! (Handshaking-Lemma)}
Nun, da die nötigen Grundlagen gelegt sind, kann es auch schon mit der ersten wichtigen Erkenntnis in der Graphentheorie, nämlich dem Handshaking-Lemma, weitergehen.\\
\bigskip
Bevor wir jedoch dazu kommen, müssen wir erneut eine relevante Definition einführen:\\
\bigskip
\begin{definition} Sei \(G = (V, E)\) ein Graph. Der \textbf{Grad} \(d(v)\) eines Knotens \(v \in V\) ist die Anzahl des Auftretens von \(v\) als Endknoten einer Kante. Wenn \(G\) einfach ist, so ist \(d(v)\) gleich der Anzahl der Nachbarn von v.
\end{definition}
\bigskip
Diese Definition lässt sich sehr einfach auf unseren Beispielgraphen vom Anfang übertragen: Um beispielsweise den Grad des Knotens "Katrin" zu bestimmen, zählen wir einfach, wie viele Kanten von dem Knoten ausgehen, also, mit wie vielen Studierenden Katrin aus der Gruppe von 4 Menschen befreundet ist, also die Anzahl der Nachbarn des Knoten "Katrin". In diesem Falle sind es \(3\) Kanten, also ist Katrin mit 3 Studierenden (Tom, Erik, Lisa) befreundet.\\
Das Zählen der Freunde funktioniert aber nur, da G einfach ist. Würde es beispielsweise noch eine Schlinge für "Studierender x kennt sich selbst" geben, so wäre der Knotengrad jeweils die Anzahl der Nachbarn der Knoten \(+1\).\\
\bigskip
Weiterhin bezeichnen wir für einen Knoten \(v \in V\):
\begin{itemize}
	\item \(\delta(G)\) als \textit{Minimalgrad} eines Graphen, und
	\item \(\Delta(G)\) als \textit{Maximalgrad} eines Graphen.
\end{itemize}
\textit{Frage an den geneigten Leser.}\\
Was sind jeweils der Minimal- und Maximalgrad des Beispielgraphen?\\
\bigskip
Nun können wir auch schon zur versprochenen wichtigen Erkenntnis kommen: Dem Handshaking-Lemma.\\
\begin{lemma}
Die Summe der Knotengrade entspricht zweimal der Anzahl der Knoten, also
\begin{equation*}
	\sum_{v \in V}d(v) = 2 \cdot |E| = 2m.
\end{equation*}
\end{lemma}
\bigskip
\begin{proof}{Beweis.}
	Um das Handshaking-Lemma zu beweisen, nehmen wir uns einen endlichen einfachen Graphen und zählen die inzidierenden Paare $(v, e)$ auf jeweils zwei verschiedene Arten und Weisen, um beide Seiten der Gleichung zu erhalten.\\
	Schauen wir uns zunächst die linke Seite der Gleichung, also $\sum_{v \in V}d(v)$ an. Die Knotengrade $deg(v)$ bilden die Anzahl der Kanten, mit denen die Knoten $v$ jeweils inzidieren. Um sich dies besser zu verdeutlichen, schauen wir uns einmal unseren Beispielgraphen vom Anfang an, jedoch mit der Veränderung, dass Katrin, Tom, Erik und Lisa sich diesmal zu viert verabreden, um sich alle untereinander kennenzulernen. In Falle eines solchen Treffens würden alle vier sich zu Beginn die Hand untereinander geben. Wenn man die Handschläge in einem Graphen darstellen möchte, dann wäre jeder Knoten mit jedem anderen Knoten durch eine Kante verbunden ($deg(v)$ wäre hier also für alle $v$ $3$: jeder Knoten inzidiert jeweils mit $3$ Kanten). Damit zählt $\sum_{v \in V}deg(v)$ also die inzidierenden Paare $(v, e)$ für alle $v$ im Graphen.\\
	Nun zur linken Seite der Gleichung, also $2 \cdot |E|$. Diese erhält man ganz einfach, indem man sich überlegt, dass jede Kante im Graphen zu genau zwei inzidierenden Paaren gehört, also zu zwei unterschiedlichen Knoten $v_1, v_2$. Dadurch erhält man für jede Kante $e$ des Graphen jeweils die inzidierenden Paare $(v_1, e)$ und $(v_2, e)$. Verdeutlicht am Beispiel bedeutet dies: Zu jedem Handschlag $(e)$ gehören zwei Leute $(v_1, v_2)$ dazu. Also erhält man damit die linke Seite der Gleichung.\\
	Da nun beide Seiten jeweils die gleiche Menge zählen, nämlich die inzidierenden Paare $(v, e)$, erhält man Gleichheit.
\end{proof}
\bigskip
Nun, da wir dieses doch recht anschauliche Lemma bewiesen haben, folgt sogleich noch eine weitere grundlegende Erkenntnis aus der Graphentheorie:\\
\begin{lemma}
	In einem Graphen ist die Anzahl der Knoten mit \emph{ungeradem Grad} gerade.\\
\end{lemma}
\bigskip
Dies mag zunächst etwas technisch wirken, ist jedoch ebenfalls recht einfach zu beweisen:
\begin{proof}
	%%%%TODO: Beweis einfügen%%%%
\end{proof}
\newpage
\section{Jeder kennt jeden -- vollständige Graphen}
Jetzt, mit etwas Grundwissen gewappnet, können wir uns einem weiteren wichtigen Begriff aus der Graphentheorie widmen: den \emph{vollständigen Graphen}.\\
\begin{definition}
	Ein Graph $G = (V, E)$ heißt \emph{vollständig}, wenn jeder Knoten zu jedem anderen benachbart ist.
	\begin{itemize}
		\item[$\rightarrow$] Da $G$ nur von $n$ abhängt, bezeichnen wir $G$ als $K_n$.
		\item[$\rightarrow$] Es gilt: $\delta(G) = \Delta(G)$.
	\end{itemize}
\end{definition}
\bigskip
Diese Definition lässt sich leicht veranschaulichen: Im Beispielgraphen vom Anfang erkennt man, dass dies kein vollständiger Graph ist, da sich die vier Studierenden nicht alle untereinander kennen. Ändert man das Beispiel jedoch so ab, wie es beim Beweis des Handshaking-Lemmas getan wurde, erhält man einen vollständigen Graphen: Jeder gibt jedem die Hand, jeder lernt die anderen aus der Gruppe kennen; jeder Knoten ist zu jedem anderen benachbart.\\
\bigskip
In diesem Zusammenhang können wir nun einen weiteren Begriff aus dem Reich der vollständigen Graphen kennenlernen -- nämlich \emph{bipartite Graphen}.\\
\begin{definition}
	Ein Graph $G = (V, E)$ heißt \emph{bipartit}, wenn seine Knoten in zwei Teilmengen an Knoten $A, B \in V$ so zerlegt werden können, dass nur Kanten \emph{zwischen}, aber nicht \emph{innerhalb} einer Knotenmenge verlaufen.
	\begin{itemize}
		\item[$\rightarrow$] Falls ein Graph die \emph{maximale Anzahl an Kanten} $m_{\mathrm{max}} = |A| \cdot |B|$ besitzt, nennt man ihn \emph{vollständig bipartit}.
	\end{itemize}
\end{definition}
\bigskip
\emph{Frage an den geneigten Leser.}\\
Ist unser Beispielgraph bipartit? Wäre er es, wenn er vollständig wäre?\\
\newpage
\section{Was sind Wege?}
Was sind Wege? - eine einfache Frage, würde man sich vielleicht denken, jedoch sind mathematische Wege etwas anderes als das, was wir umgangssprachlich als "Weg" bezeichnen. Bevor wir diese Frage sauber mathematisch beantworten können, müssen wir jedoch zunächst noch etwas Vorarbeit leisten - beginnend mit Teilgraphen:\\
\begin{definition}
	Sei $G = (V, E)$ ein Graph. Ein Graph $G' = (V', E')$ ist ein \emph{Teilgraph} von $G$, wenn $V' \subseteq V$ und $E' \subseteq E$ gilt.
		\par\bigskip
	Falls $E'$ jede Kante aus $E$ enthält, die zwei Knoten aus $V'$ verbindet, so sagt man: $G'$ ist ein von $V' \subseteq V$ \emph{induzierter Teilgraph} bzw.\ \emph{Untergraph} von $G$.
\end{definition}
Diese Begriffe sind eigentlich selbsterklärend und genau das, wonach sie klingen - quasi Teilmengen des jeweiligen Graphen.\\
\bigskip
\emph{Frage an den geneigten Leser.}\\
Was sind die Teilgraphen unseres Beispielgraphen am Anfang? Welches sind die induzierten Teilgraphen?\\
\bigskip
Nun kommen wir zur nächsten Definition, welche wir ebenfalls noch als Vorarbeit benötigen, um schließlich den Begriff des Weges einzuführen:\\
\begin{definition}
	Ein \emph{Kantenzug $Z$} in $G$ ist eine Folge von Knoten und Kanten in $G$, die sich wie folgt hintereinanderschreiben lässt:
	 \begin{equation*}
		Z = e_0e_1e_2 \ldots e_k
	\end{equation*}
	 wobei $e_i = \{x_i, x_{i+1}\}$ gilt ($x_i \in V$ sind hier die Knoten).
\end{definition}
\bigskip
Ein Kantenzug ist also ebenfalls das, wonach es klingt - man schreibt einfach verschiedene "benachbarte" Kanten hintereinander auf. Dabei ist es wichtig zu erwähnen, dass hierbei gleiche Kanten mehrfach durchlaufen werden dürfen, und Start- und Endknoten eines Kantenzuges sind egal.\\
\begin{small}
	\begin{emph}
		Es ist anzumerken, dass der Kantenzug alternativ auch noch so definiert werden kann:\\
		\begin{equation*}
			Z = x_0e_0x_1e_1x_2e_2 \ldots x_{k-1}e_{k-1}x_k
		\end{equation*}
		wobei $e_i = \{x_1, x_{i+1}\}$ gilt. (Die Definition läuft also über die Knoten anstatt die Kanten.)
	\end{emph}
\end{small}
\bigskip
Nun sind wir soweit, dass wir endlich einen mathematischen Weg definieren können:\\
\begin{definition}
	Sei $Z$ ein Kantenzug. Definiere nun $Z$ als
	\begin{itemize}
		\item einen \emph{Weg}: Alle Kanten in $Z$ sind verschieden. Die Knoten $x_0$ und $x_k$ sind die \emph{Endknoten} eines Weges $p$.
		\item einen \emph{Kreis}: Ein \emph{Kreis} ist ein Weg, bei dem gilt: $x_0 = x_k$ (gleicher Anfangs- und Endknoten).
	\end{itemize}
\end{definition}
\bigskip
Ein Weg ist also ein Kantenzug, der zwar die gleichen Knoten durchlaufen darf, jedoch keine gleichen Kanten. Bei einem Kreis dürfen ebenfalls gleiche Knoten durchlaufen werden, und Anfangs- und Endknoten müssen zwingend gleich sein.\\
\bigskip
%%%TODO: Ist das richtig?
%%%\emph{Frage an den geneigten Leser.}\\
%%%In welcher Art von (endlichen) Graphen lässt sich immer ein Kreis finden?\\
Nun, da wir um diese wichtigen Definitionen wissen, gibt es noch eine \emph{einfache} Sache zu beachten;\\
\begin{definition}
	Wir definieren...
	\begin{itemize}
		\item \emph{einfacher Weg:} Ein Weg heißt \emph{einfach}, wenn keine doppelten Knoten vorkommen, also kein Knoten doppelt "`durchlaufen"' wird.
		\item \emph{einfacher Kreis:} Ein Kreis heißt \emph{einfach}, wenn außer dem Endknoten kein Knoten doppelt vorkommt.
	\end{itemize}
\end{definition}
%%%%TODO: Irgendwelche kreativen Beispiele einfügen?
Kommen wir jetzt zu einer sehr machtvollen Erkenntnis über einfache Wege und einfache Kreise:\\
\begin{theorem}
	Jeder Graph $G$ enthält einen einfachen Weg der Länge $\delta(G)$. Falls $\delta(G) \geq 2$ gilt, dann enthält er auch einen einfachen Kreis von mindestens der Länge $\delta(G) + 1$. ($\delta(G)$ ist der Minimalgrad.)
\end{theorem}
Wollen wir dies nun beweisen.\\
\begin{proof}
	%%%%TODO: blablabla Beweis blablabla
\end{proof}
Dies gibt uns die mächtige Erkenntnis, dass wir in jedem (endlichen, einfachen) Graphen einen einfachen Weg und in jedem Graphen mit $\delta(G) \geq 2$ sogar einen einfachen Kreis finden können!\\
\bigskip
Zu guter letzt wollen wir uns in diesem Kapitel noch eine Definition anschauen, welche im weiteren Verlauf der Graphentheorie sehr wichtig werden wird: Zusammenhängende Graphen.\\
\begin{definition}
	Ein Graph $G$ heißt \emph{zusammenhängend}, wenn zu je zwei Knoten $u, v \in V$ von $G$ ein \emph{Weg} von $u$ nach $v$ existiert.
	\par\bigskip
	Nicht zusammenhängende Graphen heißen \emph{unzusammenhängend}.
	\par\bigskip
	Eine \emph{Zusammenhangskomponente} $H$ von $G$ ist ein \emph{maximal zusammenhängender Teilgraph} von $G$.
\end{definition}
\bigskip
\emph{Frage an den geneigten Leser.}\\
Finde einen zusammenhängenden Graphen und stelle diesen graphisch dar.\\

\newpage
\section[Wozu das Ganze?]{Wozu das Ganze? (Ausblick, Anwendungen)}
\end{document}
